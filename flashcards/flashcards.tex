%%%%%%%%%%%%%%%%%%%%%%%%%%%%%%%%%%%%%%%%%%%%%%%%%%%%%%%%%%%%%%%%%%%%%%%%%%%%%%%%
%% Fred Brockstedt
% Lernkarten fuer Stochastik I 
% HU-Berlin 2013

\documentclass[avery5388,grid,frame]{flashcards}

\cardfrontstyle[\large\slshape]{headings}
\cardbackstyle{empty}

\begin{document}

\cardfrontfoot{Stochstik I (2013)}

\begin{flashcard}[Definition]{Ereigniss}
  Eine Teilmenge $A \in \Omega$ heißt ein \underline{\textit{Ereignis}}.\\\\
Wir sagen, ein Ereignis tritt ein, falls für das realisierte Elementarereignis $\omega \in \Omega$ gilt: $\omega \in A$.\\\\
i) unmögliches Ereignis: $A=\emptyset$\\\\
ii) sicheres Ereignis: $A=\Omega$\\\\
iii) $A$ tritt nicht ein $\Leftrightarrow$ $A^c=\Omega\setminus A$ tritt ein \\\\
\end{flashcard}


\end{document}
